\documentclass[modern]{aastex62}

\pdfoutput=1

\usepackage{lmodern}
\usepackage{microtype}
\usepackage{url}
\usepackage{amsmath}
\usepackage{amssymb}
\usepackage{natbib}
\usepackage{multirow}
\usepackage{graphicx}
\bibliographystyle{aasjournal}

% Matrix fix:
% http://tex.stackexchange.com/questions/317824/letter-c-appearing-inside-pmatrix-environment-with-aastex
\makeatletter
\def\env@matrix{\hskip -\arraycolsep\let\@ifnextchar\new@ifnextchar\array{*{\c@MaxMatrixCols}c}}
\makeatother

% Column spacing in matrix
% http://tex.stackexchange.com/questions/275725/adjusting-separation-between-matrix-entries
\setlength\arraycolsep{25pt}

% ------------------ %
% end of AASTeX mods %
% ------------------ %

% Projects:
\newcommand{\project}[1]{\textsf{#1}}
\newcommand{\kepler}{\project{Kepler}}
\newcommand{\ktwo}{\project{K2}}

% references to text content
\newcommand{\documentname}{\textsl{Note}}
\newcommand{\figureref}[1]{\ref{fig:#1}}
\newcommand{\Figure}[1]{Figure~\figureref{#1}}
\newcommand{\figurelabel}[1]{\label{fig:#1}}
\renewcommand{\eqref}[1]{\ref{eq:#1}}
\newcommand{\Eq}[1]{Equation~(\eqref{#1})}
\newcommand{\eq}[1]{\Eq{#1}}
\newcommand{\eqalt}[1]{Equation~\eqref{#1}}
\newcommand{\eqlabel}[1]{\label{eq:#1}}

% TODOs
\newcommand{\todo}[3]{{\color{#2}\emph{#1}: #3}}
\newcommand{\dfmtodo}[1]{\todo{DFM}{red}{#1}}
\newcommand{\alltodo}[1]{\todo{TEAM}{red}{#1}}
\newcommand{\citeme}{{\color{red}(citation needed)}}

% math
\newcommand{\T}{\ensuremath{\mathrm{T}}}
\newcommand{\dd}{\ensuremath{ \mathrm{d}}}
\newcommand{\unit}[1]{{\ensuremath{ \mathrm{#1}}}}
\newcommand{\bvec}[1]{{\ensuremath{\boldsymbol{#1}}}}
\newcommand{\Gaussian}[3]{\ensuremath{\frac{1}{|2\pi #2|^\frac{1}{2}}
            \exp\left[ -\frac{1}{2}#1^\top #2^{-1} #1 \right]}}

% VECTORS AND MATRICES USED IN THIS PAPER
\newcommand{\Normal}{\ensuremath{\mathcal{N}}}
\newcommand{\mA}{\ensuremath{\bvec{A}}}
\newcommand{\mC}{\ensuremath{\bvec{C}}}
\newcommand{\mS}{\ensuremath{\bvec{\Sigma}}}
\newcommand{\mL}{\ensuremath{\bvec{\Lambda}}}
\newcommand{\vw}{\ensuremath{\bvec{w}}}
\newcommand{\vy}{\ensuremath{\bvec{y}}}
\newcommand{\vt}{\ensuremath{\bvec{\theta}}}
\newcommand{\vm}{\ensuremath{\bvec{\mu}(\bvec{\theta})}}
\newcommand{\vre}{\ensuremath{\bvec{r}}}
\newcommand{\vh}{\ensuremath{\bvec{h}}}
\newcommand{\vk}{\ensuremath{\bvec{k}}}

% typography obsessions
\setlength{\parindent}{3.0ex}

\begin{document}\raggedbottom\sloppy\sloppypar\frenchspacing

\title{%
Gradient-based methods for the characterization of exoplanets
}

\author[0000-0002-9328-5652]{Daniel Foreman-Mackey}
\affil{Flatiron Institute, New York, NY}

\begin{abstract}

As larger and more precise datasets continue to be collected for the discovery
and characterization of exoplanets, we must also develop computationally
efficient and rigorous methods for making inferences based on these data.
The efficiency and robustness of numerical methods for probabilistic inference
can be improved by using derivatives of the model with respect to the
parameters.
In this paper, I derive methods for exploiting these gradient-based methods
when fitting exoplanet data based on radial velocity, astrometry, and
transits.
Alongside the paper, I release an open source, well-tested, and
computationally efficient implementation of these methods.
These methods can substantially reduce the computational cost of fitting large
datasets and systems with more than one exoplanet.

\end{abstract}

\keywords{%
methods: data analysis ---
methods: statistical
}

\section{Outline}

\begin{enumerate}

\item Introduction to the characterization of exoplanets and gradient-based
methods

\item Automatic differentiation and inference frameworks

\item Hamiltonian Monte Carlo and variants

\item Custom gradients required for exoplanet datasets
\begin{itemize}
\item Radial velocities and Kepler's equation
\item Astrometry
\item Transits
\end{itemize}

\item Implementation details and benchmarks

\item Examples

\item Discussion

\end{enumerate}

\section{Introduction}

This is a section.

\section{Automatic differentiation}

The main limitation of gradient-based inference methods is that, in order to
use them, you must \emph{compute the gradients}!
The fundamental quantity of interest is the first derivative (gradient) of the
log-likelihood function (or some other goodness-of-fit function) with respect
to the parameters of the model.
In all but the simplest cases, symbolically deriving this gradient function is
tedious and error-prone.
Instead, it is generally preferable to use a method (called automatic
differentiation) that can automatically compute the exact gradients of the
model \emph{to machine precision} at compile time.
I would like to emphasize the fact that I am not talking \emph{numerical}
derivatives like finite difference methods, and there is no approximation
being made.

The basic idea behind automatic differentiation is that code is always written
as the composition of basic functions and, if we know how to differentiate the
subcomponents, then we can apply the chain rule to differentiate the larger
function.
This realization was one of the key discoveries that launched the field of
machine learning called ``deep learning'', where automatic differentiation was
given the name ``backpropagation'' (CITE).
Since then, there has been a substantial research base that has made these
methods general and computationally efficient, and from this, many open source
libraries have been released that ease the use of these methods (e.g.\
TensorFlow, PyTorch, Stan, ceres, Eigen, etc.\ CITATIONS).
Despite the existence of these projects, automatic differentiation has not
been widely adopted in the astronomical literature because there is some
learning curve associated with porting existing code bases to a new language
or framework.
Furthermore, many projects in astronomy involve fitting realistic, physically
motivated models that involve numerical solutions to differential equations
or special functions that are not natively supported by the popular automatic
differentiation frameworks.

In this paper, I demonstrate how to incorporate exoplanet-specific functions
into these frameworks and derive the functions needed to characterize
exoplanets using gradient-based methods applied to radial velocity,
astrometry, and transit datasets.
Similar derivations would certainly be tractable for microlensing, direct
imaging, and other exoplanet characterization methods, but these are left to a
future paper.

\section{Limb-darkened transit light curves}

Based on \project{batman} \citep{Kreidberg:2015}.

The transit depth:
\begin{eqnarray}
\delta(r,\,z) = \sum_{n=1}^{N-1} \frac{1}{2}\,\left[I(x_{n+1};\,\theta) +
I(x_n;\,\theta)\right]\,
\left[A(x_{n+1},\,r,\,z) - A(x_n,\,r,\,z)\right]
\end{eqnarray}
\dfmtodo{(quad vs trap)}.
In this expression, $A$ is defined as
\begin{eqnarray}
A(x,\,r,\,z) = \left\{\begin{array}{ll}
x^2\,\cos^{-1}(u)+r^2\,\cos^{-1}(v)-\sqrt{w}/2 &
    \left|r-x\right| < z < r + x \\
\pi\,r^2 & x \ge r + z \\
\pi\,x^2 & r \ge x + z \\
0 & \mathrm{otherwise}
\end{array}\right.
\end{eqnarray}
where
\begin{eqnarray}
u &=& \frac{z^2+x^2-r^2}{2\,z\,x} \,,\\
v &=& \frac{z^2-x^2+r^2}{2\,z\,r} \,,\,\mathrm{and}\\
w &=& (x+r+z)\,(-x+r+z)\,(x-r+z)\,(x+r-z)
\end{eqnarray}
$A(x,\,r,\,z)$ is the area of intersection of two circles with radii $x$ and
$r$ with a center-to-center separation of $z$ \citep{Mandel:2002,
Kreidberg:2015}.

A key component of computing the gradient of the transit light curve is the
gradient of the occulted area.
After some tedious algebra, we find the following surprisingly simple results
\begin{eqnarray}
\frac{\dd A(x,\,r,\,z)}{\dd z} &=& \left\{\begin{array}{ll}
-\sqrt{w}/z &
    \left|r-x\right| < z < r + x \\
0 & \mathrm{otherwise}
\end{array}\right.\\
\frac{\dd A(x,\,r,\,z)}{\dd r} &=& \left\{\begin{array}{ll}
2\,r\,\cos^{-1}(v) & \left|r-x\right| < z < r + x \\
2\,\pi\,r & x \ge r + z \\
0 & \mathrm{otherwise}
\end{array}\right.\\
\frac{\dd A(x,\,r,\,z)}{\dd x} &=& \left\{\begin{array}{ll}
2\,x\,\cos^{-1}(u) & \left|r-x\right| < z < r + x \\
2\,\pi\,x & r \ge x + z \\
0 & \mathrm{otherwise}
\end{array}\right.
\end{eqnarray}
We won't actually use the gradient with respect to $x$, but we have included
it here for completeness.

Then the gradient expressions needed for backpropagation through the transit
depth can be computed as
\begin{eqnarray}
\frac{\dd \delta(r,\,z)}{\dd I_n} = \frac{1}{2}
    \left[A(x_{n+1},\,r,\,z) - A(x_{n-1},\,r,\,z)\right]
\end{eqnarray}
\begin{eqnarray}
\frac{\dd \delta(r,\,z)}{\dd \alpha} = \sum_{n=1}^{N-1} \frac{1}{2}\,\left[
I(x_{n+1};\,\theta) + I(x_n;\,\theta)\right]\,
\left[\frac{\dd A(x_{n+1},\,r,\,z)}{\dd\alpha} -
\frac{\dd A(x_n,\,r,\,z)}{\dd \alpha}\right]
\end{eqnarray}
where $\alpha$ indicates one of $r$ or $z$.


\bibliography{exoplanet}


\end{document}
